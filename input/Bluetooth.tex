\section{Bluetooth}

\begin{figure}[H]
\begin{center}
	\includegraphics[width=120mm]{data/Bluetooth.png}
	\caption{Grobstruktur der Bluetooth-Software} %picture caption
	\label{fig:Grobstruktur_Bluetooth}
\end{center}
\end{figure}

Obenstehende Abbildung \ref*{fig:Grobstruktur_Bluetooth} zeigt die Grobstruktur der Bluetooth-Software. Über die Software erfolgt ein Input (Software – Input), welche das Suchen von Bluetooth Signalen in der Nähe auslöst. Vom Beacon mit dem stärksten Signal wird dann die Beacon-ID empfangen. Diese Beacon-ID wird Software-intern weitergeleitet, um das zugehörige Audio-File abzuspielen. So kann die Erfüllung des Ziels 6.1 und 6.2 gewährleisted werden. Während des Abspielens der Audio-Datei ist das weitere Suchen deaktiviert, um Überschneidungen von Programmabläufen und daraus resultierende mögliche Fehler zu minimieren. Nach dem Abspielen einer Audio-Datei wird das Suchen wieder ermöglicht. Um den Raumzutritt zu kontrollieren, werden die Beacons in entsprechende Raumgruppen unterteilt, welche dann am Museumseingang von einem Mitarbeiter aktiviert oder deaktiviert werden können. Für ein mögliches Einbinden des Dojo in ein Tracking-System wird der Dojo in der Lage sein, seine eigene Erkennungsnummer auf Anfrage zu senden, ähnlich wie ein Beacon. Dies würde dem Wunschziel 2 entsprechen.
\subsection{Schnittstellen zu anderen Bereichen:}
Dojo soll, um Energie zu sparen, erst per Knopfdruck das Kunstobjekt mit der stärksten Signalstärke suchen und die entsprechende Datei dafür abspielen. Dies führt Software-intern zu einer Parameterübergabe an die Audio-/Speichersektion. Während der Audiowiedergabe soll das BT ausgeschaltet bleiben, weshalb wiederum Software-intern eine Parameterübergabe bzw. -Abfrage erfolgen muss. Dies soll verhindern, dass während der Audio-Wiedergabe das Suchen und Abspielen eines weiteren Objekts möglich ist. Beim Wunschziel 1 Daten-Austausch herrscht eine enge Verbundenheit zwischen BT und Speicher. 
