\pagebreak
\textbf{Abstract:}

In our society, the importance of electrical power consumption has been steadily increasing. In order to keep the economy growing at a constant pace, either more and more energy must be produced or the efficiency of our energy use must be increased. To reduce the power consumption of a single device, it is first necessary to know its power consumption. Therefore, a power measurement device, called P3T7, for individual single-phase devices was designed. To understand the power consumption, P3T7 records the power used over a certain period of time. The device then transmits the data to a program which processes and visualizes the data. To measure the voltage, a voltage divider circuit is used and with two instrument shunts the current measurement can be realized. The circuit for the signal preparation includes amplifiers as well as first-order filters. LT Spice was used to simulate this circuit. The pre-processed signals are then sent to a microcontroller (ATMEGA2560), where the software prepares the data and writes it on an EEPROM (Electrically Erasable Programmable Read-Only Memory). By using a Bluetooth module, the data is transmitted to a java application which visualizes the data. P3T7 can detect power up to 5W with an uncertainty of 0.5W. Up to a power of 2.3 kW, P3T7 has an uncertainty of 10W. Harmonics can be detected until 1400 Hz. Consumer safety is guaranteed due to a plastic casing which prevents the user from coming into contact with any electrical parts. In conclusion P3T7 could be an advantage for households to trace power consumption and minimize them.

\vspace{13cm}
Keywords:\\
P3T7, Single-Phase Power Measurement, Power Measurement Device, Power Consumption, Wireless Transmission






\pagebreak